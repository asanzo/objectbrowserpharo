%-----------------------------------------------------------------------------
%
%               Template for sigplanconf LaTeX Class
%
% Name:         sigplanconf-template.tex
%
% Purpose:      A template for sigplanconf.cls, which is a LaTeX 2e class
%               file for SIGPLAN conference proceedings.
%
% Author:       Paul C. Anagnostopoulos
%               Windfall Software
%               978 371-2316
%               paul@windfall.com
%
% Created:      15 February 2005
%
%-----------------------------------------------------------------------------


\documentclass{sigplanconf}

% The following \documentclass options may be useful:
%
% 10pt          To set in 10-point type instead of 9-point.
% 11pt          To set in 11-point type instead of 9-point.
% authoryear    To obtain author/year citation style instead of numeric.

\usepackage{amsmath}
\usepackage[utf8]{inputenc}

\begin{document}

\conferenceinfo{WXYZ '05}{date, City.} 
\copyrightyear{2005} 
\copyrightdata{[to be supplied]} 

\titlebanner{banner above paper title}        % These are ignored unless
\preprintfooter{short description of paper}   % 'preprint' option specified.

\title{Title Text}
\subtitle{}

\authorinfo{Carla Griggio$^\dagger$ 
			\and Germán Leiva$^{\dagger\ddagger}$ 
			\and Guillermo Polito$^{\dagger\ddagger}$  
			\and Gisela Decuzzi$^\dagger$  
			\and Nicolás Passerini$^{\dagger\ddagger}$ }
           {$^\dagger$Universidad Tecnológica Nacional 
           \hspace{2cm}$^\ddagger$Universidad Nacional de Quilmes
           \\ Argentina}
           {\{carla.griggio|giseladecuzzi|leivagerman|guillermopolito|npasserini\}@gmail.com}
% \authorinfo{}
%            {Universidad Tecnológica Nacional \\ Universidad Nacional de Quilmes \\ Argentina}
%            {\{leivagerman|guillermopolito|\\npasserini\}@gmail.com}

\maketitle

\begin{abstract}
This paper describes the features that a programming environment should have in order to help learning the object-oriented programming (OOP) paradigm and let students get the skills needed to build software using objects very quickly. This proposal is centered on providing graphical tools to help understand the concepts of the paradigm and let students create objects before they are presented the class concept \cite{Utti2010}. The object, message and reference concepts are considered of primary importance during the teaching process, allowing quick acquisition of both theory and practice of concepts such as delegation, polymorphism and composition [1].

Additionally, a current implementation of the proposed software and the experience gained so far using it for teaching at universities and work trainings. Finally, we describe possible extensions to the proposed software that are currently under study.
\end{abstract}

\category{CR-number}{subcategory}{third-level}

\terms
term1, term2

\keywords
keyword1, keyword2

\section{Introduction}

The text of the paper begins here.

\appendix
\section{Appendix Title}

This is the text of the appendix, if you need one.

\acks

Acknowledgments, if needed.

% We recommend abbrvnat bibliography style.

\bibliographystyle{abbrvnat}
\bibliography{loop}

% The bibliography should be embedded for final submission.

% \begin{thebibliography}{}
% \softraggedright
% 
% \bibitem[Smith et~al.(2009)Smith, Jones]{smith02}
% P. Q. Smith, and X. Y. Jones. ...reference text...
% 
% \end{thebibliography}

\end{document}
